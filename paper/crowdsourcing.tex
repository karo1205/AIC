% THIS IS SIGPROC-SP.TEX - VERSION 3.1
% WORKS WITH V3.2SP OF ACM_PROC_ARTICLE-SP.CLS
% APRIL 2009
%
% It is an example file showing how to use the 'acm_proc_article-sp.cls' V3.2SP
% LaTeX2e document class file for Conference Proceedings submissions.
% ----------------------------------------------------------------------------------------------------------------
% This .tex file (and associated .cls V3.2SP) *DOES NOT* produce:
%       1) The Permission Statement
%       2) The Conference (location) Info information
%       3) The Copyright Line with ACM data
%       4) Page numbering
% ---------------------------------------------------------------------------------------------------------------
% It is an example which *does* use the .bib file (from which the .bbl file
% is produced).
% REMEMBER HOWEVER: After having produced the .bbl file,
% and prior to final submission,
% you need to 'insert'  your .bbl file into your source .tex file so as to provide
% ONE 'self-contained' source file.
%
% Questions regarding SIGS should be sent to
% Adrienne Griscti ---> griscti@acm.org
%
% Questions/suggestions regarding the guidelines, .tex and .cls files, etc. to
% Gerald Murray ---> murray@hq.acm.org
%
% For tracking purposes - this is V3.1SP - APRIL 2009

\documentclass{acm_proc_article-sp}
%\usepackage{graphicx}
%\usepackage{natbib}
\usepackage[utf8]{inputenc}
%\usepackage{tabularx}
%\usepackage{hyperref}
%\usepackage{color}
%\usepackage[usenames,dvipsnames,svgnames,table]{xcolor}

\begin{document}

\title{Crowdsourcing: State of the art}
%\subtitle{}
%
% You need the command \numberofauthors to handle the 'placement
% and alignment' of the authors beneath the title.
%
% For aesthetic reasons, we recommend 'three authors at a time'
% i.e. three 'name/affiliation blocks' be placed beneath the title.
%
% NOTE: You are NOT restricted in how many 'rows' of
% "name/affiliations" may appear. We just ask that you restrict
% the number of 'columns' to three.
%
% Because of the available 'opening page real-estate'
% we ask you to refrain from putting more than six authors
% (two rows with three columns) beneath the article title.
% More than six makes the first-page appear very cluttered indeed.
%
% Use the \alignauthor commands to handle the names
% and affiliations for an 'aesthetic maximum' of six authors.
% Add names, affiliations, addresses for
% the seventh etc. author(s) as the argument for the
% \additionalauthors command.
% These 'additional authors' will be output/set for you
% without further effort on your part as the last section in
% the body of your article BEFORE References or any Appendices.

\numberofauthors{3} %  in this sample file, there are a *total*
% of EIGHT authors. SIX appear on the 'first-page' (for formatting
% reasons) and the remaining two appear in the \additionalauthors section.
%
\author{
  % You can go ahead and credit any number of authors here,
  % e.g. one 'row of three' or two rows (consisting of one row of three
  % and a second row of one, two or three).
  %
  % The command \alignauthor (no curly braces needed) should
  % precede each author name, affiliation/snail-mail address and
  % e-mail address. Additionally, tag each line of
  % affiliation/address with \affaddr, and tag the
  % e-mail address with \email.
  %
  % 1st. author
  \alignauthor
  Bernhard Gößwein\\%\titlenote{This author is one of them who did all the really hard work.}\\
        \affaddr{some really, really important information here}\\
	\affaddr{e8727334@student.tuwien.ac.at}
  % 2nd. author
  \alignauthor
  Robert Kapeller\\
        \affaddr{some really, really important information here}\\
	\affaddr{e106884@student.tuwien.ac.at}
  % 3rd. author
  \alignauthor
  David Riepl\\%\titlenote{This author is one of them who did all the really hard work.}\\
        \affaddr{some really, really important information here}\\
	\affaddr{e0625016@student.tuwien.ac.at}
  %\and  % use '\and' if you need 'another row' of author names
  % 4th. author
} %/author
\date{30 July 1999}
% Just remember to make sure that the TOTAL number of authors
% is the number that will appear on the first page PLUS the
% number that will appear in the \additionalauthors section.

\maketitle
\begin{abstract}
\textit{The paper should provide an overview about the scientific state of the art in crowdsourcing. Note
that the paper is not the documentation of your tool – it should discuss scientific papers related
to this topic in the style of a seminar paper. Good starting points for finding related scientific
papers are the sources cited in this text, Google Scholar 6 , IEEEXplorer 7 or the ACM Digital
Library 8 . Use the ACM ’tight’ conference style 9 (two columns), and keep it brief (3 pages). You
do not necessarily need to install a LaTeX environment for this - you can use writeLaTeX 10 , a
collaborative paper writing tool as well.}
\end{abstract}

\terms{Crowdsourcing, some, more terms}

\keywords{Crowdsourcing, some, more, keywords} % NOT required for Proceedings

\section{Introduction}
\textit{ The paper should provide an overview about the scientific state of the art in crowdsourcing. Note
that the paper is not the documentation of your tool – it should discuss scientific papers related
to this topic in the style of a seminar paper. Good starting points for finding related scientific
papers are the sources cited in this text, Google Scholar 6 , IEEEXplorer 7 or the ACM Digital
Library 8 . Use the ACM ’tight’ conference style 9 (two columns), and keep it brief (3 pages). You
do not necessarily need to install a LaTeX environment for this - you can use writeLaTeX 10 , a
collaborative paper writing tool as well.
}

Write something useful about how crowdsourcing has been developed over time, history and the like.
Come then to the current state of the topic and give a brief overview about the (possible) future.

%/footnote{some text in here}

\section{History of Crowdsourcing}

\section{Current use cases}
This section shall contain some of the currently used crowd sourcing applications. How is it used currently? What are the benefits/advantages compared to other methods, like algorithmic approaches? ``Is it worth it?''

\section {Current scientific works}
%Citations to articles \cite{bowman:reasoning, clark:pct, braams:babel, herlihy:methodology},
What are the current scientific areas in conjunction with crowdsourcing? 

\section{Future of Crowdsourcing}

\section{Conclusion}
Some final words about crowd sourcing in general and the future of it.


%\begin{figure}
%\centering
%\epsfig{file=fly.eps}
%\caption{A sample black and white graphic (.eps format).}
%\end{figure}


%ACKNOWLEDGMENTS are optional
\section{Acknowledgments}
This section is optional; it is a location for you
to acknowledge grants, funding, editing assistance and
what have you.  In the present case, for example, the
authors would like to thank Gerald Murray of ACM for
his help in codifying this \textit{Author's Guide}
and the \textbf{.cls} and \textbf{.tex} files that it describes.

%
% The following two commands are all you need in the
% initial runs of your .tex file to
% produce the bibliography for the citations in your paper.
\bibliographystyle{abbrv}
\bibliography{crowdsourcing}  % sigproc.bib is the name of the Bibliography in this case
% You must have a proper ".bib" file
%  and remember to run:
% latex bibtex latex latex
% to resolve all references
%
% ACM needs 'a single self-contained file'!
%
%APPENDICES are optional
%\balancecolumns
%\appendix
%Appendix A

\end{document}
